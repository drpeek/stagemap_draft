\section{Expert Interview}
\label{section:feedback}

We demonstrated how the prototype could be applied to analyze the vehicle fault sequence data to three groups of analysts from the automotive industry. The analysts all dealt with similar data in their daily work and they were very familiar with the usage scenario. One group of analysts was interviewed remotely and interacted with the system through our web server. The other two interviews were conducted face to face. For each interview, we first introduced the visual designs and the interactions in the system, and then asked analysts to explore the system on their own \revision{for about half an hour}. \revision{After that, we had discussion sessions with the domain experts focusing on three different aspects of the system, e.g., system usability, required additional features and other potential uses (besides vehicle fault analysis) of the system.} The analysts commented positively on the system and were intrigued by the idea of fuzzy pattern matching and sequence clustering. \revision{Most of the experts think that one of the most powerful features in the system is the interactive alignment of the sequence clusters.} Furthermore, one analyst commented that ``the system shows clearly the seriousness of some faults as it might later lead to other faults [based on the summary view and the detailed view]'', ``the correlation among the faults are very clear to see [in the radial graph]'' and ``with more data it would be a powerful tool to spot patterns of fault occurrences''. Seeing the great potential value of the system, the analysts have already arranged follow-up discussions with us about offering the visual analytics solution as part of their vehicle data analytics software.\looseness=-1

Besides that, the analysts also requested additional features in the system. For example, now the system only supports aligning on a single event and they recommended to generalize this feature to support aligning at two or even more events to identify what happened between those anchor points.\looseness=-1

One analyst mentioned that vehicles from many car manufacturers record error logs in the same manner. Therefore, the system could benefit different car brands. The analyst pointed out that although the current system was demonstrated with a small sample dataset, the features in the system could become more powerful with large scale data. \looseness=-1

During the demonstration we also mentioned that the algorithm and the system were generic and could be used to analyze other datasets such as website click streams/application logs as well. One analyst immediately recalled that they also collect click stream data for vehicle diagnostics software used in repair shops and suggested that ``the system can help optimize the interface, [and] shorten the time [for the repairers] to find information''. After that he/she asked for further follow-up to fully assess the feasibility of this approach and showed great interest to also continue pursuing this particular usage scenario. This demonstrated that the principle underlying the system can be easily grasped and it has the versatility to be adapted to different application scenarios.\looseness=-1

