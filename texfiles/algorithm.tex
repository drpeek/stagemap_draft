
\section{Summary of Stage Progression for Event Sequences}
\label{section:algorithm}

In this section, we introduce the proposed data model to extract the stage progression summary. We first give the problem statement and then describe the detailed algorithm to generate the summary which can be directly visualized by the system. 

\subsection{Problem Statement}

Our idea of visualizing event sequences is to extract a set of stages and summarize the data into several stage progression patterns. Stage is a frequently occurring subsequence among the sequences and usually contain high-level structures. Fig.~\ref{fig:summary}(a) shows seven event sequences. We can represent each sequence as a series of stages, as shown in Fig.~\ref{fig:summary}(b). There are five different stages, and each of them occurs at least twice in the sequences. To provide a concise and meaningful overview of stages, we can then identify a set of stage progression patterns. In Fig.~\ref{fig:summary}(c), seven sequences are splitted into two groups. The high-level progression pattern of each group can be described with a unified structure. In real-world application, such as web clickstream analysis, these patterns can reveal certain human behavior which is critical for the analysis.

In actual practices, the dataset is much more complex and noisy. To identify the user requirements, we have surveyed existing works, explored several real-world dataset and collaborated with analysts in online learning clickstream analysis. We observe that an effective visual summary of event sequences should have the following charactoristics: 

\begin{compactitem}
	\item \textbf{Minimizing visual complexity.} Different from existing pattern mining models, the proposed model should also consider the visual encoding of the summary and try to reduce visual complexity.
	\item \textbf{Dealing with individual variance.} As shown in Fig.~\ref{fig:summary}, when matching each sequence with stages, there are events which cannot fit into any stages. We define these events as addition events. Such addition events occur due to system error when recording the data or because of same behaviors still contain individual variance. Therefore, on one hand the model need to be robust to addition events to group similar sequences, on the other hand addition events should be preserved and presented to the user in the visual summary.  
	\item \textbf{Supporting interactive analysis.} When analyzing large scale datasets, level-of-detail analysis is usually required to allow users drill down to details step by step. Therefore, the model should be computational efficient and update the summary interactively.
\end{compactitem}

\subsection{Algorithm}

\begin{algorithm}
	\KwIn{sequences $\mathscr{X}=\{X_1,X_2,...,X_n\}$}
	\KwOut{stage progression trees $\mathscr{T}=\{T_1,T_2,...,T_k\}$}
	initialize $\mathscr{T}=\{\mathscr{X}\}$, stage queue $\mathscr{Q}=\{\}$, covering $\mathscr{C}=\mathscr{X}$\;
	stage candidates $\mathscr{S}=Gap-BIDE(\mathscr{X})$\;
	$\mathscr{Q}=\{sortS(\mathscr{S})\}$\;
	\While{$\mathscr{Q}\neq\emptyset$} {
		pop $S_i$ from \mathscr{Q}\;
		\For{each $C_{j}\in\mathscr{C}$}{
			$C_j^\ast=Cover(C_j, S_i)$\;	
			replace $C_j$ with $C_j^\ast$\;
		}
		$\mathscr{T^\ast}=BuildTrees(\mathscr{C})$\;
		\If{$Cost(\mathscr{T^\ast})<Cost(\mathscr{T})$}{
			$\mathscr{T}=\mathscr{T^\ast}$\;
		}
	}
	\Return $\mathscr{T}$
	\caption{SPTree}
	\label{algorithm:pipeline}
\end{algorithm} 

The algorithm for summarizing progression patterns is called \textit{SPTree}. Algorithm~\ref{algorithm:pipeline} describes the whole algorithm pipeline. Taken the input sequences $\mathscr{X}$, \textit{SPTree} first generates a set of stage candidates $\mathscr{S}$ which occur frequently in the dataset. To this end, we employ \textit{Gap-BIDE}~\cite{li2008efficiently}, a technique used for efficiently mining closed subsequences. We employ this technique mainly because it introduces wildcards when matching potential subsequences, which is consistent with our idea of allowing individual variance in each sequence. To be more specific, for each stage candidates, we can set up a constraint $Th_{gap}$ in \textit{Gap-BIDE} to limit the maximum number of wildcards when matching sequences. Then, \textit{Gap-BIDE} will identify all the subsequences which have the frequency higher than a predefined threshold $Th_{sup}$. The candidates will then be used to construct the summary. 

The algorithm then iteratively updates the summary by adding more stages from the candidates. We sort the candidates so that more frequently occurred stages can be considered first. This is the core subroutine of \textit{SPTree}. Each iteration can be separated into two phases, that is, updating covering and constructing summary. In the first stage, a covering with the newly added stage is generated. Here a covering refers a presentation of a sequence with stages and additions. The covering determines where a stage matches to the original sequence. Algorithm~\ref{algorithm:cover} describes how we compute the covering. For each stage candidate $S_i$, the subroutine \textit{Cover} looks for all occurrences of $S_i$ and replaces the corresponding subsequences. To accelerate the algorithm, the matchings between stages and sequences can be stored as a lookup table when employing \textit{Gap-BIDE}, so \textit{Cover} only need to check cells in the table. 

In the second stage, \textit{SPTree} builds a summary given the updated covering. In this paper, we try to build a visual summary which highlights the branching patterns between stages. Therefore, we construct each pattern as a tree structure with each stage as a node. When constructing other structures for summarizing progression patterns, we only need to change the algorithm in the second phase (i.e., line 11 in Algorithm~\ref{algorithm:pipeline}). 

\begin{algorithm}
	\KwIn{a covering $C$, stage candidates $\mathscr{S}$}
	\KwOut{an updated covering $C^\ast$}
	\For{each $S_i\in\mathscr{S}$}{
		\For{all $o\in$ occurrences of $S_i$}{
			\If{$C \cap o=\emptyset$}{
				$C=C \cup o$\;
			}
		}	
		\If{no additions in $C$}{
			\textbf{Break}\;
		}
	}
	\Return $C$
	\caption{Cover}
	\label{algorithm:cover}
\end{algorithm}

\textit{Gap-BIDE} can generate a rich set of candidates, however, some of the candidates are not valid for an optimized summarization. To validate each candidate, inspired by~\cite{chen2018sequence}, we further introduce a cost function as described in Eqn.~\ref{eq:cost}:

\begin{equation}
Cost(\mathscr{T},\mathscr{C}) = \sum_{S \in \mathscr{S}, T \in \mathscr{T}}\{1| S \in T\} + \lambda||addition||
\label{eq:cost}
\end{equation}

The two terms in Eqn.~\ref{eq:cost} correspond to the main visual elements in the summary, that is, the number of nodes in trees and the number of addition events. The parameter $\lambda$ is introduced to balance the importance of the two terms. At each iteration, the summary $\mathscr{T}$ is updated only when the candidate can reduce the cost.  

\textbf{Computational complexity.} In \textit{SPTree}, the computational complexities for \textit{Gap-BIDE}, \textit{Cover} and \textit{BuildTrees} are [TODO], $O(m)$, $\sum_{S \in \mathscr{S}, T \in \mathscr{T}}\{1| S \in T\}$ respectively, where $m$ is the number of stage candidates. The most time consuming step is generating stage candidates. However, this step can be computed offline without interfering interactive analysis. For example, when users select a subset of sequences, e.g., sequences belong to a specific progression pattern, the alogorithm can update the candidates by directly recalculate the threshold of stages with a time complexity of $O(mn)$, where $n$ is the number of sequences. We report the running time of the proposed algorithm in Section~\ref{section:evaluation}.

\textbf{Parameter settings.} There are three parameters $Th_{sup}$, $Th_{gap}$, $\lambda$ need to be predefined. $Th_{sup}$ ranges from 0 to $n$. It should be set small enough so that no important stages are eliminated at the first step. Based on experiments, we set $Th_{sup}$ to be 0.05 times $n$ since the result becomes steady when further decreases the thresould. $Th_{gap}$ ranges from 0 to the maximum length of sequences. Cross validation can be used to select the optimized value. $\lambda$ is used to balance the importance of patterns and additions. [TODO] 


