
Temporal event sequences, i.e., ordered series of events which occurred over time, appear in a wide range of domains such as website click streams, user interaction logs in software applications, electronic health records and car service records. Analyzing such data can help yield meaningful insights and therefore support decision making. For example, by analyzing user interaction logs, designers can identify those users who confront usability issues and then design interventions to improve user experience. Similarly, by analyzing the online learning click streams, instructors can better understand the learning behavior of students and improve the course design.  

Real world event sequences are often complex. A typical dataset can contain thousands or more distinct sequences with hundreds distinct events. The length of each sequences can vary from a few to hundreds of events. Therefore, many existing visualization techniques are inadequate to directly present the data. Instead of visualizing the raw data, recent works try to visualize the summary of the data with the help of pattern mining models~\cite{stasko2000focus+,polack2015timestitch,perer2014frequence,kwon2016peekquence,liu2017patterns,wang2016unsupervised,liu2017coreflow,guo2018eventthread}. These methods have shown promising results since the data summary has a much lower visual complexity. However, extracting and providing a scalable and meaningful visual summary is still challenging due to the following reasons: First, the existing visual summaries are still not concise enough for large scale dataset. It is desirable to have an overview which itself can support level-of-detail analysis and become even more scalable. Second, traditional pattern mining models only preserve or prioritize statistically significant events and therefore have the risk of misleading users on domain specific tasks. So the summary should keep the detailed information and make sure users are aware of the individual variance within the summary. 

In this paper, we propose StageMap, an event sequence summarization method which tries to present sequences with a set of stage progression patterns. Stage progression patterns can be found in many event sequence dataset. For example, in online learning click streams analysis, when a student tries to finish an online assignment, he/she may first browse the assignment and then review a course material or ask questions on the course forum. Since many students follow the same steps to finish the assignment, these sequence of learning activities can be defined as a stage while each activity is recorded as an event. The whole learning behavior of a student can be modeled as a progression of various stages. The benefits of presenting sequences with stages are two-fold: First, since the stage itself can be considered as a summary of events, stage-based summary is in general more concise compared with event-based summary. So it can handle more complex dataset, especially when the length of sequences vary a lot. Second, in many applications, stages contain high-level semantics, so users can easily understand the progression pattern without digging into detailed events.  

Our method first extracts a set of frequently occurred stages. We support soft pattern match when extracting stages so that later the individual variance is allowed in the stage progression patterns. We then develop an algorithm to transform the original sequences into a set of progression patterns. Each pattern represents a group of similar sequences and how the corresponding stages evolve over time. As in the online learning click streams analysis, a pattern can show a group of students who share similar learning behavior and show how they gradually learn different topics and finish various course activities. In this paper, a pattern is modeled as a tree structure while each node in the tree is a stage. Other structure such as directed graph can also be applied to present the pattern for different application scenarios. The proposed algorithm can also preserve the hierarchical structure of the sequences so that each pattern can split into several detailed patterns. We then design a visual analytics system to support visual analysis of the summary. The system has three linked views: the stage map view to show the summary of identified progression patterns, the tree view to represent the detailed events for a highlighted pattern and the sequence view to visualize each individual sequences. We further test our method on two real world datasets. 

To summarize, the main contribution of this work include: 
\begin{compactitem}
	\item A summarized representation of event sequences with a set of stage progression trees as well as an algorithm to transform raw sequences into the summary.
	\item An interactive visual analytics system which allows users to explore the summary of the data.
	\item Case studies with real world datasets to demonstrate the effectiveness of the approach.
\end{compactitem}




